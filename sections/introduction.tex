\section{Introduction}

Declarative languages permit application logic to be specified in a high-level manner to suit human designers and maintainers, while, automatically compiling down to low-level primitives that suit modern compute resources, permitting extraction of parallelism. These systems automatically bridge human-oriented problem specification with machine-oriented parallel implementation -- making it possible for non-HPC users to write high performance applications.
Datalog~\cite{Aref:2015, ceri1989you, de2012datalog, huang2011datalog, ullman1983principles} a declarative logic programming language is an example of such a framework -- it is a lightweight deductive database system where queries and database updates are expressed as first-order Horn-clause rule. Running a datalog program creates (deduces) the intensional database (output) which extends data from the extensional database (input) with all data transitively derivable via the program’s rules. Datalog programs can be \emph{implemented} using relational algebra (RA).
RA comprises an important basis of operations on whole relations including join, projection, aggregation, selection, that transform one or more input relation into an output relation.
Standard operations on relations such as selection, join, and union may be used in combination to implement efficient kernels that infer new facts from available facts in a (fixed-point) loop. 

A scalable high-performance relational algebra system has the potential to automatically extract massive data-parallelism from applications built on top of declarative languages such as Datalog~\cite{Aref:2015, ceri1989you, de2012datalog, huang2011datalog, ullman1983principles}. Datalog has been applied in a wide variety of applications, including in bioinformatics~\cite{king2004applying}, big-data analytics~\cite{halperin2014demonstration, seo2013socialite, shkapsky2016big}, natural language processing~\cite{mooney1996inductive}, networking and distributed systems~\cite{alvaro2010dedalus, conway2012logic, loo2009declarative}, program understanding~\cite{hajiyev2006codequest}, and program analysis~\cite{bravenboer2009strictly, lam2005context, smaragdakis2013set}. Despite embodying great expressive power, and being a subject of enduring interest in databases research, the high-performance computing (HPC) community has thus far produced quite limited explorations of parallel RA on supercomputers~\cite{balkesen2013multi, Cacace:1991:OPS:111828.111831, Cheiney:1990:PST:94362.94445, kim2009sort, Valduriez:1988:PET:54616.54618}. To date, RA has not received remotely the same attention as other foundational operations in HPC such as parallelizing linear algebra or stencil computations.